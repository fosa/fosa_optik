%coding:utf-8

%----------------------------------------
%FOSAPHY, a LaTeX-Code for a summary of optics
%Copyright (C) 2014, Mario Felder, Michael Fallegger

%This program is free software; you can redistribute it and/or
%modify it under the terms of the GNU General Public License
%as published by the Free Software Foundation; either version 2
%of the License, or (at your option) any later version.

%This program is distributed in the hope that it will be useful,
%but WITHOUT ANY WARRANTY; without even the implied warranty of
%MERCHANTABILITY or FITNESS FOR A PARTICULAR PURPOSE.  See the
%GNU General Public License for more details.
%----------------------------------------

\chapter{Quanten Optik}
\section{Konstanten}
\[
\boxed{\begin{aligned}	
		&h = 6.62606872\cdot 10^{-34} J\cdot s \\
		&m_e = 9.1094\cdot 10^{-31}kg \\
		&eV = 1.602176565 \cdot 10^{-19}J\\
	\end{aligned}}	\]
\\
\section{Grundformeln}
Umrechnung:
\[
	E_J \cdot 1.602\cdot 10^{-19}= E_{eV}
\]
\begin{footnotesize}
	$\mu$: Längenmasse
\end{footnotesize}
\\
\section{Photon- das Lichtquantum}
Die maximale kinetische Energie nimmt linear mit der Frequenz des Lichts zu. Es wird kein Photostrom beobachtet für $f<f_0$.\
\[
	E=h\cdot f = \hbar \cdot \omega = \frac{h \cdot c}{\lambda}
\]
\begin{footnotesize}
	$h$:	Planck'sche Konstante = $6.62606872\cdot 10^{-34} J\cdot s$\\
\end{footnotesize}
\begin{center}
	\includegraphics[scale = 0.3]{images/photon_energie.jpg}
\end{center}
\
\\
\section{Impuls des Photons - Compton Streuung}
Obwohl das Photon keine Ruhemasse besitzt, hat es trotzdem einen Impuls. Die Relativitätstheorie liefert für das Photon mit Wellenlänge $\lambda$ den Impuls $p$:\

\[
	p=\frac{E}{c}=\frac{h\cdot f}{c}= \frac{h}{\lambda} \\ \text{pro Photon} \\ \\
\]
Energie- und Impulserhaltung liefern:
\[
	\Delta\lambda = \lambda^` -\lambda=\frac{h}{m_e\cdot c}\cdot \left( 1- \cos \phi  \right) = \lambda_c \cdot \left( 1- \cos \phi   \right) )
\]
\begin{center}
	\includegraphics[scale = 0.3]{images/impuls_photon.jpg}
\end{center}
\
\section{Bremsstrahlung}
In einer Röntgenröhre werden Elektronen von einer geheizten  Kathode auf die Anode beschleunigt. Maximale Frequenz (kleinste Wellenlänge) kann ein Bremstrahlungs-Photon erreichen, wenn es die ganze kinetische Energie des Elektrons schluckt.\
\[
	q_e\cdot V_{AC} = h\cdot f_{max} = \frac{h\cdot c}{\lambda_{max}}=\frac{\Delta E}{h\cdot c}
\]
\\
\section{Energieniveaus von Wasserstoff}
Balmer-Formel:
\[
	\frac{1}{\lambda}=R_y\cdot \left( \frac{1}{2^2} - \frac{1}{n^2}\right) 
\]
\begin{footnotesize}
	$R_y$=	$1.097\cdot 10^7 m^{-1}$\\
	$n$=	$3,4,5,...$ \\
\end{footnotesize}
\[
	E_n= -\frac{h\cdot c \cdot R_y}{n^2}
\]
\begin{footnotesize}
	$R_y$=	$1.097\cdot 10^7 m^{-1}$\\
	$n$=	$1,2,3,...$ \\
\end{footnotesize}
\begin{center}
	\includegraphics[scale = 0.25]{images/en_H.jpg}
\end{center}
\
\subsection{Bohr's Herleitung der Balmerformel}
Wasserstoff:
\[
	E_n= - \frac{1}{\varepsilon_0^2}\cdot\frac{m_e\cdot q_e^4}{8n^2\cdot h^2}
\]
Helium+:
\[
	E_n= - \frac{1}{\varepsilon_0^2}\cdot\frac{m_e\cdot q_e^4}{2n^2\cdot h^2}
\]


